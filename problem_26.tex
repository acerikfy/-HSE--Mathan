\documentclass[a4paper,12pt]{article}

\usepackage[T1,T2A]{fontenc}        % Кодировки шрифтов
\usepackage[utf8]{inputenc}         % Кодировка текста
\usepackage[english,russian]{babel} % Подключение поддержки языков

\usepackage{amsthm}                 % Оформление теорем
\usepackage{amstext}                % Текстовые вставки в формулы
\usepackage{amsfonts}               % Математические шрифты
\usepackage{amsmath}

\newtheorem*{ther}{Теорема}
\newtheorem*{defi}{Определение}
\newtheorem*{task}{Задачка}

\newcommand{\eps}{\varepsilon}

\begin{document}

    \section*{Билет 26}
    
    Локальные свойства непрерывных функций (об ограниченности, о сохранение знака) и докажите одно из них (на выбор экзаменатора)

    \textbf{Локальные свойства непрерывных функций:}
        \begin{enumerate}
            
            \item Если функция f непрерывна в точке a, то она ограниченна в некоторой окрестности этой точки :
            $$\exists c>0   \exists U_\delta(a) :$$ 
            $$\forall x \in {U_\delta(a)} : |f(x)| < c$$ 
            
            Следует из свойств пределов.
            
            \item Если функция f непрерывна в точке $a$ и $f(a)\neq  0$, то в некоторой окрестности точки a знак функции совпадает со знаком числа f(a):
            $$\exists U_\delta(a) : \forall x \in {U_\delta(a)} \rightarrow sign f(x)=sign f(a) $$
            
            Следует из свойств пределов.
            
            \item Если $f$ и $g$ непрерывны в точке $a$, то функции :
            $f \pm g$ , $f\cdot g$ , $\frac{f}{g}$  непрерывны в точке a.
            
            Следует из непрерывности и свойств пределов.
            
            \item Если $z=f(y)$ непрерывна в точке $y$, а $y=\varphi(x)$  , непрерывна в точке $x_0$ причем $y_0=\varphi(x_0)$  , то в некоторой окрестности $x_0$  определена сложная функция равная $f[\varphi(x)]$  которая также непрерывна в точке $x_0 $:
                       $$ \lim\limits_{y\to y_0}f(y)=f(y_0) \vee \lim\limits_{x\to x_0}\varphi(x)=\varphi(x_0)$$
                    $$\Rightarrow \lim\limits_{x\to x_0}f[\varphi(x)]=f[\varphi(x_0)] $$  

            Композиция непрерывных функций также является непрерывной.

        \end{enumerate}

        \begin{task}
            Найти значения $a$ и $b$, при которых функция

            \begin{equation*}
                y = 
                 \begin{cases}
                   \frac{3\cdot 2^{x - 1} - 3 ^ x}{\sqrt{x + 3} - 2} &\text{$x < 1$}\\
                   ax + b, &\text{$1 \leq x \leq 2$}\\
                   \frac{sin(\pi x)}{\cos(\frac{\pi} {4} x)}  &\text{$2 \leq x$}\\
                 \end{cases}
            \end{equation*}

            непрерывна на промежутке $[0; 4]$
        \end{task}
        \begin{proof}[Решение]
            \textbf{(1)} На $[0;\ 1]$ и $(2;\ 4)$ $f(x)$ непрерывна (т.к. в них $\lim\limits_{x\to a-0} f(x)= \lim\limits_{x\to a + 0}` f(x)$ и $\exists f(a)$)

            \textbf{(2)} $f(x)$ -- непрерына в т. $1$ $\Rightarrow$
            $$\lim\limits_{x\to 1+0} f(x) = a + b = \lim\limits_{x\to 1-0} f(x) = $$

            $$ \lim\limits_{x\to 1-0} 
                \frac{3\cdot ((e^{ln(2)(x -1)} - 1) - (e^{ln(3)(x -1)} - 1)) \cdot (\sqrt{x+3} + 2)} 
                     {x + 3 - 4} =
            $$

            $$
                = \lim\limits_{x\to 1-0}\frac{3\cdot (x - 1)(ln(2) - ln(3)) \cdot (\sqrt{x+3} + 2)} 
                     {x - 1} = $$
            $$
            =  \lim\limits_{x\to 1-0} 3\cdot (x - 1)(ln(2) - ln(3)) \cdot (\sqrt{x+3} + 2) = 12 (ln2 - ln3)
            $$

            $$\Downarrow$$
            $$a + b  = 12 (ln2 - ln3)$$

            \textbf{(3)}
            $f(x)$ непрерывна в т. 2 $\Rightarrow$ 

            $$\lim\limits_{x\to 2-0} f(x) = 2a + b = \lim\limits_{x\to 2+0}f(x)$$

            $$\lim\limits_{x\to 2+0} f(x) = 
            \lim\limits_{x\to 2+0} \frac{sin(\pi x)}{cos(\frac{\pi}{4}x)} = \lim\limits_{y\to 0+0} = \frac{sin(\pi y)}{cos(\frac{\pi}{4}y + \frac{\pi}{2})} = *$$

            Пусть $y = x - 2$. Тогда $y \to 0+0$, $x = y + 2$.
            $$ * = \lim\limits_{y\to 0+0} \frac{sin(\pi y)}{-sin(\frac{\pi}{4}y)}  = \lim\limits_{y\to 0+0} -\frac{sin(\pi y) \cdot \frac{\pi}{4}y}
                        {\pi y \cdot sin(\frac{\pi}{4} y)}\cdot \frac{4\pi}{\pi} = -4
            $$

            $$\Rightarrow 2a + b = -4$$

            И, хвала небесам, получаем:
            \begin{equation*}
                 \begin{cases}
                   a + b = 12 (ln2 - ln3) \\
                   2a + b = -4
                 \end{cases}
            \end{equation*}

            \begin{equation*}
                 \begin{cases}
                   a = -4 - 12 (ln2 - ln3) \\
                   b = 24(ln2 - ln3) + 4
                 \end{cases}
            \end{equation*}

            \textbf{Ответ:} \begin{equation*}
                 \begin{cases}
                   a = -4 - 12 (ln2 - ln3) \\
                   b = 24(ln2 - ln3) + 4
                 \end{cases}
            \end{equation*}
        \end{proof}
\end{document}